% Copyright 2004 by Till Tantau <tantau@users.sourceforge.net>.
%
% In principle, this file can be redistributed and/or modified under
% the terms of the GNU Public License, version 2.
%
% However, this file is supposed to be a template to be modified
% for your own needs. For this reason, if you use this file as a
% template and not specifically distribute it as part of a another
% package/program, I grant the extra permission to freely copy and
% modify this file as you see fit and even to delete this copyright
% notice. 

\documentclass{beamer}
% Replace the \documentclass declaration above
% with the following two lines to typeset your 
% lecture notes as a handout:
%\documentclass{article}
%\usepackage{beamerarticle}


% There are many different themes available for Beamer. A comprehensive
% list with examples is given here:
% http://deic.uab.es/~iblanes/beamer_gallery/index_by_theme.html
% You can uncomment the themes below if you would like to use a different
% one:
%\usetheme{AnnArbor}
%\usetheme{Antibes}
%\usetheme{Bergen}
%\usetheme{Berkeley}
%\usetheme{Berlin}
%\usetheme{Boadilla}
%\usetheme{boxes}
%\usetheme{CambridgeUS}
%\usetheme{Copenhagen}
%\usetheme{Darmstadt}
%\usetheme{default}
%\usetheme{Frankfurt}
\usetheme{Goettingen}
%\usetheme{Hannover}
%\usetheme{Ilmenau}
%\usetheme{JuanLesPins}
%\usetheme{Luebeck}
%\usetheme{Madrid}
%\usetheme{Malmoe}
%\usetheme{Marburg}
%\usetheme{Montpellier}
%\usetheme{PaloAlto}
%\usetheme{Pittsburgh}
%\usetheme{Rochester}
%\usetheme{Singapore}
%\usetheme{Szeged}
%\usetheme{Warsaw}

\title{Primer on Text editting and more with GNU Emacs}

% A subtitle is optional and this may be deleted
\subtitle{3. Introduction to Scientific Computing with Linux\\Part II. Basic Tools}

\author{Nidish Narayanaa B\inst{1}}% \and S.~Another\inst{2}}
% - Give the names in the same order as the appear in the paper.
% - Use the \inst{?} command only if the authors have different
%   affiliation.

\institute[Universities of Somewhere and Elsewhere] % (optional, but mostly needed)
{
  \inst{1}%
  Department of Aerospace Engineering\\
  Indian Institute of Space Science \& Technology, Trivandrum
  }
% - Use the \inst command only if there are several affiliations.
% - Keep it simple, no one is interested in your street address.

\date{FOSS Club, IIST, 2016}
% - Either use conference name or its abbreviation.
% - Not really informative to the audience, more for people (including
%   yourself) who are reading the slides online

\subject{EmacsTut_IISTFOSS}
% This is only inserted into the PDF information catalog. Can be left
% out. 

% If you have a file called "university-logo-filename.xxx", where xxx
% is a graphic format that can be processed by latex or pdflatex,
% resp., then you can add a logo as follows:

% \pgfdeclareimage[height=0.5cm]{university-logo}{university-logo-filename}
% \logo{\pgfuseimage{university-logo}}

% Delete this, if you do not want the table of contents to pop up at
% the beginning of each subsection:
\AtBeginSubsection[]
{
  \begin{frame}<beamer>{Outline}
    \tableofcontents[currentsection,currentsubsection]
  \end{frame}
}

% Let's get started
\begin{document}

\begin{frame}
  \titlepage
\end{frame}

\begin{frame}{Outline}
  \tableofcontents
  % You might wish to add the option [pausesections]
\end{frame}

% Section and subsections will appear in the presentation overview
% and table of contents.
\section{Introduction}

\subsection{Text Editting - IDEs, Text Editors and Emacs}

\begin{frame}{Text Editting}{IDEs, Text Editors and Emacs}
  \begin{itemize}
  \item {
	As the need for computation grows, need for a comfortable workspace grows too
  }
  \item {
	We present here the "advanced, self-
documenting, customizable, extensible editor Emacs"\footnote{Richard Stallman, \emph{GNU Emacs Manual}, Ed 16, 2010 }
  }
  \item {
  	It is one of the oldest and highly developed text editors out there
  	}
  \item {
  	Innumerable text editors and more importantly, IDEs (for programmers) are out there - but why Emacs ?
  	}
  \end{itemize}
\end{frame}

\subsection{Some Flashy Stuff}
\begin{frame}{Some flashy stuff}{Why Emacs?}
\begin{block}{Tagline}<1->
Advanced, Self-Documenting, Customizable and Extensible
\end{block}
\begin{itemize}
\item<2-> {
\emph{Advanced}, since Emacs can do much more than simple insertion and deletion - from automatic indentation and split screen modes to controlling subprocesses 
}
\item<3-> {
\emph{Self-Documenting}, since there is an exhaustive documentation repository that comes with Emacs, which is expanded with each package that the user installs
}
\item<4-> {
\emph{Customizable}, since the behavior of commands can easily be altered
}
\item<5-> {
\emph{Extensible}, since you can go beyond simple customization and create your own commands
}
\end{itemize}
\end{frame}

\begin{frame}{Some Flashy Stuff}{How Emacs?}
\begin{block}{Newsflash}<1->
If you've not realized it already, you're almost always faster with your hands on the keyboard and off that wretched mouse ;)\\\uncover<2-> {Imagine productivity when you can stop rubbing on that thing!}
\end{block}
\begin{itemize}
\item<3-> Emacs works with "commands" which are a sequence of keys with a preceding \textbf{C- (Command key; ctrl)} or an \textbf{M- (Meta ket; alt)}
\item<3-> These commands are usually the shortcuts to pre-defined functions. \uncover<4->{Soon we shall learn how to create our own functions and assign commands}
\item<5-> Emacs also doesn't come pre-loaded with all the features. \uncover<6->{Most of the advanced features are organized as separate \textbf{packages}}
\end{itemize}
\end{frame}

\section{The Nitty-Gritties}
\subsection{Basics}
\subsubsection{Installation}
% You can reveal the parts of a slide one at a time
% with the \pause command:
\begin{frame}{Installation}
\begin{block}{Distribution}
Although there are a lot of Emacs distributions, we highly recommend the \textbf{GNU Emacs} for Linux users and \textbf{Aquamax} for Mac users and the door for Windows users ;)
\end{block}
\begin{block}{Install}
In ubuntu-based Linux distros, go to the terminal and type :
\begin{quote}
sudo apt-get install emacs24
\end{quote}
In Mac, make sure homebrew is available and go to the terminal and type,
\begin{quote}
brew install aquamax
\end{quote}

\end{block}
\end{frame}

\subsubsection{Basic Commands}
\begin{frame}{Basic Commands}{The Primer}
\begin{block}{CONTROL and META keys}<1->
The two keys that enable emacs to distinguish a command from the user's text. \uncover<2->{By default, C and M are mapped to the \textbf{ctrl} and \textbf{alt} keys}
\end{block}
\begin{block}{The Emacs Lexicography}<3->
\begin{itemize}
\item<4->{$C-<chr>$ and $M-<chr>$ signify that we have to press the $<chr>$ character key while holding the CONTROL/META keys respectively}
\item<5->{$M-<chr1>\,<chr2>$ signify that we have to press $<chr1>$ along with the META key and then press $<chr2>$ separately}
\end{itemize}
\end{block}
\end{frame}

\begin{frame}{Basic Commands}{Getting to Move around}
\begin{block}{Moving around}
Although I find myself comfortable with the arrow keys, the following are the inbuilt commands
\end{block}
\begin{description}
\item[C-f and C-b] Move \textbf{f}orward or \textbf{b}ackward by a letter
\pause
\item[M-f and M-b] Move \textbf{f}orward or \textbf{b}ackward by a word
\pause
\item[C-n and C-p] Move to \textbf{n}ext or \textbf{p}revious line
\pause
\item[M-v and C-v] Move up or down pages
\pause
\item[C-a and C-e] Move to beginning or end of line
\pause
\item[M-a and M-e] Move to beginning or end of sentence
\pause
\item[C-l] Toggle where the text at the cursor should be on the screen (middle-top-bottom)
\pause
\item[C-/] Undo action
\pause
\item[C-u \#] Prefix this to any command and that command is executed "\#" number of times
\pause
\item[C-g] Stop executing commands\footnote{may have to hit it twice or thrice}
\end{description}
\end{frame}

\begin{frame}{Basic Commands}{Text Editting}
\begin{description}
\item[C-d] Delete character
\pause
\item[M-d] Delete word from current point to end
\pause
\item[C-S-spc] Set mark to select text\footnote{S - Shift; spc - spacebar}
\pause
\item[C-k] Kill selected 
\pause
\item[C-w] Cut selected
\pause
\item[C-y] Yank, ie, Paste selected\footnote{deletions will not save the characters in the clipboard}
\pause
\item[C-s] Search for a regular expression (C-s to go to next occurrence; C-r to go to previous occurrence)
\item[M-\%] Query replace a regular expression

\end{description}

\end{frame}

\begin{frame}{Basic Commands}{Files and Buffers}
\begin{description}
\item[C-x C-f] Open a file - type the location and filename in the minibuffer
\pause
\item[C-x C-s] Save changes to the current file
\pause
\item[C-x C-c] End current session
\end{description}
\pause
\begin{block}{Buffers}
In Emacs, each file is opened in a buffer, which always exists in the background - Following are some commands to access and modify them
\end{block}
\pause
\begin{description}
\item[C-x b] Switch buffer - type the name of the buffer (if it's a file the name is the same as the filename) \uncover{Of course - it's got tab completion}
\pause
\item[C-x C-b] List all the buffers
\pause
\item[C-x k] Kill current buffer
\end{description}
\end{frame}

\begin{frame}{Basic Commands}{Multiple Windows and Modes}
\begin{description}
\item[C-x 2] Create a new window horizontally
\item[C-x 3] Create a new window vertically
\item[C-x \{] Decrease width of current window (M-x shrink-window-horizontally)
\item[C-x \}] Increase width of current window (M-x enlarge-window-horizontally)
\item[C-x ] Increase height of current window (M-x enlarge-window)
\end{description}
\begin{block}{Modes}
Emacs works in different modes - with each mode having its own set of commands apart from the above standard ones
\end{block}
\begin{description}
\item[M-x c-mode] Mode for the C programming language
\item[M-x shell-mode] Mode for shell-scripting
\item[M-x gnuplot-mode] 
..
\end{description}
\end{frame}

\begin{frame}{Basic Commands}{Accessing Help}
\begin{block}{Help Documentation}
Emacs comes with a vast help documentation\\
It even has its own "Starter's" tutorial - access with "C-h t" - Highly recommended!
\end{block}
\begin{description}
\item[C-h ?] Display all the help options
\item[C-h c ---] Get help about a specific command
\item[C-h f ---] Get help about a specific function
\item[C-h i] Read included manuals
\end{description}

\end{frame}

\subsection{Intermediate}
\begin{frame}{Intermediate Stuff}{Programming}
\begin{block}{Program Compilation}
If you have a makefile in the current directory, then using "C-x c" you can call the make command from inside emacs\\
A separate window will be opened for you to view the compilation message.
\end{block}
\begin{block}{Shell Mode}
Accessed with "M-x shell-mode", an instance of the shell is presented in a window inside emacs. Your experience will be some where between editting a text file and using the terminal. This, used as a second split screen, can be used to edit and execute a program side-by-side
\end{block}
\end{frame}

\begin{frame}{Intermediate Stuff}{Programming}
\begin{block}{Debugging with gdb}
Although I personally prefer \emph{ddd} for interactive debugging, emacs has got a complete gdb environment, similar to that of gdb - with interactive code and variable displays
\end{block}
\begin{block}{Autocompletion}
Add the following to your ~/.emacs file:
\begin{quote}
(require\,'package)\\
(add-to-list 'package-archives '("melpa" . "https://melpa.org/packages/"))\\
(when ($<$ emacs-major-version 24)\\
(add-to-list 'package-archives '("gnu" . "http://elpa.gnu.org/packages/")))
\end{quote}
\end{block}
\end{frame}

\begin{frame}{Intermediate Stuff}{Programming}
\begin{block}{Autocompletion}
After this list all the packages (from the archives) with "M-x package-list-packages" and search for the auto-complete-mode packages and install the required ones either by selecting it or by typing "M-x package-install $<$package-name$>$". You will now be able to see the corresponding files in the ~/.emacs.d directory.
\end{block}
\end{frame}

\begin{frame}{Intermediate Stuff}{User-defined commands}
\begin{block}{Init File}
The file ~/.emacs can be used to store commands that are to be called at the startup of each emacs session. These are in a flavor of LISP called Elisp.
\end{block}
\begin{block}{A heads up to Lisp}
\begin{itemize}
\item All LISP statements are enclosed in parantheses - ( statement ) - get used to looking at such code
\item The basic format of a lisp program is,
$$(\,<function\,name>\,var1\,var2\,var3\,...\,)$$
\item Each statement has a return value; for example, y=sin(x) must be written as,
$$(setq\,y\,(sin\,x))$$
\end{itemize}
\end{block}
\end{frame}

\begin{frame}{Intermediate Stuff}{A heads up to Lisp}
\begin{quote}
Lisp is not Case-Sensitive
\end{quote}
\begin{block}{Variable and Function Declarations}
\begin{description}
\item[Variable] \emph{defvar} for variable declaration,
\begin{quote}
(defvar \emph{x} 2)\\
(defvar \emph{x} (read))
\end{quote}

The former declares a variable \emph{x} with value 2 and the latter declares a variable \emph{x} which will take value from stdin. To change the value of an already existing variable use (setq \emph{x} newval).
\item[Function] \emph{defun} for function declaration,
\begin{quote}
(defun\,fname\,(*arg*)\\(format t "$\sim$a!\,$\sim$\%"\,*arg*))
\end{quote}
The function may be called by (fname var).
\end{description}
\end{block}
\end{frame}

\begin{frame}{Intermediate Stuff}{Back to Emacs}
\begin{block}{Setting modes at startup}
To enable/disable a mode at startup, say the menu-bar-mode (which displays a menu bar when enabled), use
\begin{quote}
(menu-bar-mode -1) - to disable\\
(menu-bar-mode 1) - to enable
\end{quote}
\end{block}
\begin{block}{Creating a new keybinding (Command)}
This can be done in two ways.\\
Suppose I want the shell to open up in the current window with "M-s RET" (RET denotes return - enter key), I would add,
\begin{quote}
(global-set-key "\textbackslash M-s RET" 'shell)\\
or\\
(global-set-key (kbd "M-s RET") 'shell)
\end{quote}
\end{block}
\end{frame}

\subsection{Org-Mode}
\begin{frame}{The Org-Mode}{A goto tool for cataloging and tracking your activities}
\begin{block}{Org-Mode}<1->
An org file is saved as .org and you can use it to track your status in a project in a systematic fashion
\end{block}
\begin{block}{Headlines}<2->
Prefixed with one or more "*"s these denote sections and subsections based on the number of "*", with the least having the highest preference
\begin{quote}
* Section 1\\
** Subsection 1\\
** Subsection 2\\
* Section 2\\
\end{quote}
Use Tab and S-Tab to toggle subtree folding\\
M-RET to insert new heading with same level as current\\
There are other commands to shift the subtree up/down, etc
\end{block}
\end{frame}

\begin{frame}{The Org-Mode}{A goto tool for cataloging and tracking your activities}
\begin{block}{Lists}
\begin{description}
\item<1->[Ordered Lists] These may begin with 1), 2) .. or 1., 2., .. 
\item<2->[Unordered Lists] These may begin with + - 
\item<3->[Description Lists] Use :: to separate a text from its description
\end{description}
\end{block}
\begin{block}{Tables}<4->
Type the entries of the first row and hit "C-c RET" and the setup becomes a table automatically. Lines starting with "$|$-" are taken to be horizontal separators automatically - just hit "Tab" or "C-c C-c" to make it auto-indent at any point. The first row will have to be typed as,
\begin{quote}
$|$ name $|$ place $|$ animal $|$ things $|$\\
$|$-$<TAB>$\\
\end{quote}
\end{block}
\end{frame}
% Placing a * after \section means it will not show in the
% outline or table of contents.
\section*{Summary}

\begin{frame}{Summary}
  \begin{itemize}
  \item
    We went through a few \alert{basic commands} in emacs to try and familiarize ourselves with the editor
  \item
    Then we looked at the \alert{basics of Lisp} to understand and better use the back end of Emacs
  \item
	Finally we looked at the Org-mode
  \end{itemize}
  
  \begin{itemize}
  \item
    Outlook
    \begin{itemize}
    \item
	  What we have here seen is just the tip of the ice berg - we learn more with time in tools like Emacs
    \item
      Please do keep in mind that this discussion is in no way meant to be exhaustive in any of the topics covered
    \end{itemize}
  \end{itemize}
\end{frame}



% All of the following is optional and typically not needed. 
\appendix
\section<presentation>*{\appendixname}
\subsection<presentation>*{For Further Reading}

\begin{frame}[allowframebreaks]
  \frametitle<presentation>{For Further Reading}
    
  \begin{thebibliography}{10}
    
  \beamertemplatebookbibitems
  % Start with overview books.

  \bibitem{Richard Stallman}
    R.M. Stallman
    \newblock {\em GNU Emacs Manual}.
    \newblock Free Software Foundation, 2010.
    
  \bibitem{Carsten Dominik}
	C. Dominik
    \newblock {\em The compact Org-mode Guide}.
    \newblock Free Software Foundation, 2010.    
 
  \bibitem{Carsten Dominik}
	C. Dominik
    \newblock {\em The Org Manual}.
    \newblock Free Software Foundation, 2004.     

  \end{thebibliography}
\end{frame}

\end{document}


